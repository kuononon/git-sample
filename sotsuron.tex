\documentclass[paper=a4paper,fontsize=10pt]{jlreq}
\usepackage{luatexja-fontspec}
\setmainfont{Harano Aji Mincho}
\setsansfont{Harano Aji Gothic}
\setmainjfont{Harano Aji Mincho}
\setsansjfont{Harano Aji Gothic}

\usepackage{amsmath,amssymb}
\usepackage{unicode-math}
\setmathfont{LatinModernMath-Regular}

\begin{document}

\title{MnZnGd系磁性材料の合成と特性評価}
\author{三浦 玖遠}
\date{\today}
\maketitle

\tableofcontents

\part{1. 諸元}
\section{1. 諸元}
\subsection{研究背景}
\subsubsection{ナノテクノロジー}
ナノテクノロジーは、物質をナノメートル(10億分の1メートル)スケールで操作・制御する技術です。\\
この技術により、材料の特性を向上させたり、新しい機能を持つ製品を開発したりすることが可能となります。  \\
ナノテクノロジーは、医療、エレクトロニクス、エネルギー、環境など、さまざまな分野で応用されています。
\paragraph{これは段落}
\subparagraph{これこそが小段落}

\section{2. 理論}
\subsection{磁性体}
\subsubsection{磁性の種類}
磁性とは、物質が磁場に対して示す反応のことを指します。\\
磁性には主に以下の種類があります。\\
\begin{itemize}
  \item 強磁性: 鉄、コバルト、ニッケルなどの物質が持つ磁性で、外部磁場がなくても磁化を維持します。
  \item 反磁性: 一部の物質が持つ磁性で、外部磁場に対して反発する性質を持ちます。
  \item 常磁性: 一部の物質が持つ磁性で、外部磁場に対して引き寄せられる性質を持ちます。
\end{itemize}


\section{数式も埋め込めます}
$y = x$みたいな感じで、行中に埋め込めますし、以下のように書くこともできます。\\

\begin{equation}
  \int_{a}^{a}f\left(x\right)dx = 0
\end{equation}

複数行の数式も書けます。$=$の位置を揃えることもできます。\\

\begin{align}
  \int_{1}^{2}\left(x^2 + 3x\right)dx + \int_{1}^{2}\left(x^2 - 3x\right)dx &= \int_{1}^{2}\left\{\left(x^2 + 3x\right) + \left(x^2 - 3x\right)\right\}dx \\
  &= \int_{1}^{2}2x^2dx \\
  &= 2\left[\frac{x^3}{3}\right]^2_1 = \frac{2\left(2^3 - 1^3\right)}{3} = \frac{14}{3}
\end{align}

                             


\end{document}
